% cup product
\NewCommandCopy{\oldcup}{\cup}
\renewcommand{\cup}{\smile}
% in exchange...
\newcommand{\union}{\oldcup}

% sphere (spectrum)
\NewCommandCopy{\sect}{\S}
\renewcommand{\S}{\mathbb{S}}

% wedge sums use the \vee symbol, and smash products use the \wedge symbol
% exchange these
% \smash is used in subsections somehow, so we'll keep it around
\NewCommandCopy{\oldsmash}{\smash}
\NewCommandCopy{\oldwedge}{\wedge}

% wedge sum
\renewcommand{\wedge}{\vee}
% smash product
% WARNING: this is very hacky. you cannot take relative smash products like this
% If someone knows a workaround that lets you do A \smash_X B then feel free to
% contact me...
\renewcommand{\smash}[2][]{%
    \ifmmode%
        % use the wedge symbol if in math mode
        \oldwedge {#1}%
    \else%
        % if not, it's probably the subsection thing from before, so use the old smash
        \oldsmash[#1]{#2}%
    \fi%
}
% here's your relative smash product, that you surely use so much...
\newcommand{\smashrel}[1]{\oldwedge_{#1}}

% big versions
\NewCommandCopy{\oldbigwedge}{\bigwedge}
\newcommand{\bigsmash}{\oldbigwedge}
\renewcommand{\bigwedge}{\bigvee}